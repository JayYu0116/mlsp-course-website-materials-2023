\section{Source Separation using ICA}

%\subsection{Synthetic case}

In this problem, you will implement your own version of ICA and apply it to source separation.\\
You are given 2 audio recordings, \texttt{sample1.wav} and \texttt{sample2.wav}. These can be found in the directory
\texttt{hw2\_materials\_f20/problem2.}

These recordings were generated by mixing two different audio signals. Your objective is to reconstruct the original sounds using ICA. Do the following steps:
\begin{enumerate}

    \item Implement your own version of ICA based on FOBI (Freeing Fourth Moments), the method that we discussed in class. Write a function, \texttt{ica}, which receives as input a $2 \times N$ matrix and outputs a $2 \times N$ matrix where its rows are the extracted independent components. \ul{Submit your code}.
    
    \item Read the file \texttt{sample1.wav} and extract the audio signal \texttt{s1}. This should be a vector with 132,203 components. Read the file \texttt{sample2.wav} obtaining the signal \texttt{s2}. Both \texttt{s1} and \texttt{s2} have the same size. Transpose and concatenate these signals generating a matrix ${\bf M}$ with 2 rows and 132,203 columns.\\
    Apply the function \texttt{ica} on the matrix ${\bf M}$ and use the Matlab function \texttt{audiowrite} to save the components generated as \texttt{source1.wav} and \texttt{source2.wav}, respectively. Don't forget, ICA does not consider scale factors, so you may need to boost or decrease the resulting signal. \ul{Submit files \texttt{source1.wav} and \texttt{source2.wav}.}
    \item If ${\bf H}$ is a $2 \times N$ matrix where its rows correspond to the output of \texttt{ica}, then we can say that
    $$ {\bf M} = A {\bf H}$$
    In this case, $A$ is the mixing matrix which produces our observation ${\bf M}$. Compute the $2 \times 2$ mixing matrix for this case. \ul{Submit $A$ as \texttt{mixing\_matrix.csv}.} 
\end{enumerate}




